% Credits are indicated where needed. The general idea is based on a template by Vel (vel@LaTeXTemplates.com) and Frits Wenneker.

\documentclass[11pt, a4paper]{article} % General settings in the beginning (defines the document class of your paper)
% 11pt = is the font size
% A4 is the paper size
% “article” is your document class

%----------------------------------------------------------------------------------------
%	Packages
%----------------------------------------------------------------------------------------

% Necessary
\usepackage[german,english]{babel} % English and German language 
\usepackage{hyperref} % URL
\usepackage{subfig} %two figure
\usepackage{listings} % Insert code
\usepackage{float} % figure at the correct position
\usepackage{lipsum}
\usepackage{mwe}
\usepackage{booktabs} % Horizontal rules in tables 
% For generating tables, use “LaTeX” online generator (https://www.tablesgenerator.com)
\usepackage{comment} % Necessary to comment several paragraphs at once
\usepackage[utf8]{inputenc} % Required for international characters
\usepackage[T1]{fontenc} % Required for output font encoding for international characters

% Might be helpful
\usepackage{amsmath,amsfonts,amsthm} % Math packages which might be useful for equations
\usepackage{tikz} % For tikz figures (to draw arrow diagrams, see a guide how to use them)
\usepackage{tikz-cd}
\usetikzlibrary{positioning,arrows} % Adding libraries for arrows
\usetikzlibrary{decorations.pathreplacing} % Adding libraries for decorations and paths
\usepackage{tikzsymbols} % For amazing symbols ;) https://mirror.hmc.edu/ctan/graphics/pgf/contrib/tikzsymbols/tikzsymbols.pdf 
\usepackage{blindtext} % To add some blind text in your paper


%---------------------------------------------------------------------------------
% Additional settings
%---------------------------------------------------------------------------------

%---------------------------------------------------------------------------------
% Define your margins
\usepackage{geometry} % Necessary package for defining margins

\geometry{
	top=2cm, % Defines top margin
	bottom=2cm, % Defines bottom margin
	left=2.2cm, % Defines left margin
	right=2.2cm, % Defines right margin
	includehead, % Includes space for a header
	%includefoot, % Includes space for a footer
	%showframe, % Uncomment if you want to show how it looks on the page 
}

\setlength{\parindent}{15pt} % Adjust to set you indent globally 

%---------------------------------------------------------------------------------
% Define your spacing
\usepackage{setspace} % Required for spacing
% Two options:
\linespread{1.5}
%\onehalfspacing % one-half-spacing linespread

%----------------------------------------------------------------------------------------
% Define your fonts
\usepackage[T1]{fontenc} % Output font encoding for international characters
\usepackage[utf8]{inputenc} % Required for inputting international characters

\usepackage{XCharter} % Use the XCharter font


%---------------------------------------------------------------------------------
% Define your headers and footers

\usepackage{fancyhdr} % Package is needed to define header and footer
\pagestyle{fancy} % Allows you to customize the headers and footers

%\renewcommand{\sectionmark}[1]{\markboth{#1}{}} % Removes the section number from the header when \leftmark is used

% Headers
\lhead{} % Define left header
\chead{\textit{}} % Define center header - e.g. add your paper title
\rhead{} % Define right header

% Footers
\lfoot{} % Define left footer
\cfoot{\footnotesize \thepage} % Define center footer
\rfoot{ } % Define right footer

%---------------------------------------------------------------------------------
%	Add information on bibliography
\usepackage{natbib} % Use natbib for citing
\usepackage{har2nat} % Allows to use harvard package with natbib https://mirror.reismil.ch/CTAN/macros/latex/contrib/har2nat/har2nat.pdf

% For citing with natbib, you may want to use this reference sheet: 
% http://merkel.texture.rocks/Latex/natbib.php

%---------------------------------------------------------------------------------
% Add field for signature (Reference: https://tex.stackexchange.com/questions/35942/how-to-create-a-signature-date-page)
\newcommand{\signature}[2][5cm]{%
  \begin{tabular}{@{}p{#1}@{}}
    #2 \\[2\normalbaselineskip] \hrule \\[0pt]
    {\small \textit{Signature}} \\[2\normalbaselineskip] \hrule \\[0pt]
    {\small \textit{Place, Date}}
  \end{tabular}
}
%---------------------------------------------------------------------------------
%	General information
%---------------------------------------------------------------------------------
\title{Deep Learning HW3 Report} % Adds your title
%\author
%{
%    \name{Alfons Hwu} % Add your first and last name
%    %\thanks{} % Adds a footnote to your title
%    \\
%    \institution{0416324, Dept of Computer Science NCTU} % Adds your institution
%}



%---------------------------------------------------------------------------------
%	Define what’s in your document
%---------------------------------------------------------------------------------
\begin{document}


% If you want a cover page, uncomment "%---------------------------------------------------------------------------------
% Cover page
%---------------------------------------------------------------------------------

% Here are more templates for other cover pages: https://www.latextemplates.com/cat/title-pages

% This example is based on this cover page example: https://www.latextemplates.com/template/academic-title-page

\begin{titlepage} % Starts new environment where the page number is not displayed and the count starts at 1 for the next page

%------------------------------------------------
%	Institutional information
%------------------------------------------------
	
\begin{minipage}{0.4\textwidth} % Begins new environment (like a text box)
    \begin{flushleft} % Sets environment on the left side of the paper
    \large
    National Chiao Tung University\\ % Add your institution
    Spring 2019 \\ % Add term
    Deep Learning \\ % Add course title
    Instructor: Jen-Tsung Chien\\ % Add instructor/supervisor name 
    \end{flushleft}
\end{minipage}
	
\vspace*{2in} % Adds some space in-between
	
\center % Centre everything on the page

%------------------------------------------------
%	Main part
%------------------------------------------------
	
{\huge\bfseries Deep Learning Final Project Report}\\[0.4cm] % Add your paper title {\large\today}\\[0.4cm] % Add date (current day)
{\Large Neural Style Transfer to Make Photo Younger}
\\{\large Group 13 members}
\\
Alfons Hwu (Group leader) \\ % Add your name
0416324\\
Pin-Jen Hunag \\ % Add your name
0416033\\
Chia-Yu Sun \\ % Add your name
0416045\\
Min-Xue Yang\\ % Add your name
0416022\\
\vfill % Adds additional space

%------------------------------------------------
%	General information about the author
%------------------------------------------------

\vfill % Adds additional space

Dept. of Computer Science \\ % Add info about your program
Composed with \LaTeX  on Overleaf

\vfill % Adds additional space

%------------------------------------------------
%	Word count
%------------------------------------------------

\vfill % Adds additional space
	
\end{titlepage}" and uncomment "\begin{comment}" and "\end{comment}" to comment the following lines
%---------------------------------------------------------------------------------
% Cover page
%---------------------------------------------------------------------------------

% Here are more templates for other cover pages: https://www.latextemplates.com/cat/title-pages

% This example is based on this cover page example: https://www.latextemplates.com/template/academic-title-page

\begin{titlepage} % Starts new environment where the page number is not displayed and the count starts at 1 for the next page

%------------------------------------------------
%	Institutional information
%------------------------------------------------
	
\begin{minipage}{0.4\textwidth} % Begins new environment (like a text box)
    \begin{flushleft} % Sets environment on the left side of the paper
    \large
    National Chiao Tung University\\ % Add your institution
    Spring 2019 \\ % Add term
    Deep Learning \\ % Add course title
    Instructor: Jen-Tsung Chien\\ % Add instructor/supervisor name 
    \end{flushleft}
\end{minipage}
	
\vspace*{2in} % Adds some space in-between
	
\center % Centre everything on the page

%------------------------------------------------
%	Main part
%------------------------------------------------
	
{\huge\bfseries Deep Learning Final Project Report}\\[0.4cm] % Add your paper title {\large\today}\\[0.4cm] % Add date (current day)
{\Large Neural Style Transfer to Make Photo Younger}
\\{\large Group 13 members}
\\
Alfons Hwu (Group leader) \\ % Add your name
0416324\\
Pin-Jen Hunag \\ % Add your name
0416033\\
Chia-Yu Sun \\ % Add your name
0416045\\
Min-Xue Yang\\ % Add your name
0416022\\
\vfill % Adds additional space

%------------------------------------------------
%	General information about the author
%------------------------------------------------

\vfill % Adds additional space

Dept. of Computer Science \\ % Add info about your program
Composed with \LaTeX  on Overleaf

\vfill % Adds additional space

%------------------------------------------------
%	Word count
%------------------------------------------------

\vfill % Adds additional space
	
\end{titlepage}

\date{May 31, 2019}
\begin{comment}
\end{comment}
\maketitle{} %Print your title, author name and date; comment if you want a cover page 

%----------------------------------------------------------------------------------------
% Introduction
%----------------------------------------------------------------------------------------
\setcounter{page}{1} % Sets counter of page to 1

\section{Self-designed VAE for image reconstruction and generating(unsupervised learning)} % Add a section title
\subsection{Preprocessing the images} % Section check OK 20190502
In the preprocessing part, I merely resize the image to 64 x 64 and without normalization. 
\\ If normalization is used, the reconstructed / randomly generated image will look very dark and black.
\\ CNN is used in this problem since it \textbf{fits better for image processing} while DNN falls short in this part because it processes 3 color channels at the same time, causing the result image somehow looked grey(The value of R G B are similar)
\\ The CNN layers are used in the feature extraction and reconstruction part as the following code shows.
\begin{lstlisting}[language = python]
def __init__(self, image_channels = 3, h_dim = 1024, z_dim = 32):
        super(VAE, self).__init__()
        self.encoder = nn.Sequential(
            nn.Conv2d(image_channels, 32, kernel_size = 4, stride = 2),
            nn.ReLU(),
            nn.Conv2d(32, 64, kernel_size = 4, stride = 2),
            nn.ReLU(),
            nn.Conv2d(64, 128, kernel_size = 4, stride = 2),
            nn.ReLU(),
            nn.Conv2d(128, 256, kernel_size = 4, stride = 2),
            nn.ReLU(),
            flatten()
        )
        self.fc1 = nn.Linear(h_dim, z_dim)
        self.fc2 = nn.Linear(h_dim, z_dim)
        self.fc3 = nn.Linear(z_dim, h_dim)
        # The code of decoder part is just deconvolution, i.e. nn.ConvTranspose2d with reverse order and I/O dimension
        )
\end{lstlisting}
\\ Between the feature extraction and image reconstruction lies the reparameterization part, that is by multiplying $\sigma$ and shifting $\mu$, making $N~(0, 1)$ to approximate $N~(\mu, \sigma^2)$
\begin{figure}[H]
  \centering
  \subfloat[VAE ovaerall architecture]{\includegraphics[scale = 0.3]{HW3/vae/vae_arch.jpg}\label{fig:f1}}
  \hfill
  \subfloat[Reparameterization to approximate $N~(0, 1)$]{\includegraphics[scale = 0.3]{HW3/vae/reparam.jpg}\label{fig:f2}}
  \hfill
  \subfloat[Mathematical explanation]{\includegraphics[scale = 0.2]{HW3/vae/vae_idea.jpg}\label{fig:f2}}
  \hfill
  \caption{VAE explanation}
\end{figure}
\\ Image source credit to \url{https://zhuanlan.zhihu.com/p/34998569}

\subsection{Learning curve and reconstructed images}
\begin{figure}[H]
    \centering
    \subfloat[Learning curve]{\includegraphics[scale = 0.9]{HW3/vae_result/lc.png}\label{fig:f1}}
    \hfill
    \subfloat[Reconstructed image]{\includegraphics[scale = 0.3]{HW3/vae_result/re.png}\label{fig:f2}}
    \hfill
    \subfloat[Ground truth]{\includegraphics[scale = 0.3]{HW3/vae_result/gt.png}\label{fig:f2}}
    \hfill
    \caption{VAE result evaluation}
\end{figure}
\section{Self-designed cGAN for style transformation(unsupervised learning)}
\end{document}